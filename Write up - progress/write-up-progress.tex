\documentclass[11pt]{article}
\usepackage[a4paper, total={6in, 8in}]{geometry}
\usepackage{amsmath}
\usepackage{amsthm}
\usepackage{graphicx}
\usepackage{amsfonts}
\usepackage{xcolor}
\usepackage{enumerate}
\begin{document}

\newtheorem{theorem}{Theorem}
\numberwithin{theorem}{section}
\theoremstyle{definition}
\newtheorem{definition}{Definition}
\newtheorem{proposition}{Proposition}
\newtheorem{example}{Example}
\newtheorem{lemma}{Lemma}
\newtheorem{corollary}{Corollary}
\numberwithin{definition}{section}
\numberwithin{proposition}{section}
\numberwithin{example}{section}
\numberwithin{lemma}{section}
\numberwithin{corollary}{section}
\newcommand{\uw}{\mathcal{U}(W,X)}
\newcommand{\W}{$(W,S)$}
\newcommand{\ix}{\textit}
\newcommand{\tr}{\textcolor{red}}
\newcommand{\sg}{$\Sigma$}


%\title{Buildings}
%\maketitle


\section{Progress on the question}


\subsection{Statistics on positive folds}

We now restrict to looking at Weyl chamber orientations over affine Coxeter complexes. This means that we have a complex \sg, with a boundary $\partial$\sg, and that our orientations are induced by a boundary chamber orientation. Here, we can get a partial answer to our main question of calculating the shadow of a given gallery. To do this, we define a $\phi$-valuation map on our set of alcoves. We can then prove a recursive algorithm for calculating the shadow of a gallery.

First, given a gallery, we want to calculate the number of positive folds of this gallery that we can make. A proof of this proposition can be found in \cite{DEL}.

\begin{proposition}
    Consider the largest element $w_0$ in $W_0$. Given an $x\in W$, and a $\phi$-postive (multi)folding $\gamma$ of $\gamma_x$, we have
    \[l_R(xy^{-1})\leq |F(\gamma)|\leq l(w_0),\]
    where $y:=\tau($ft$(\gamma))$.
\end{proposition}


\begin{definition}
    Let $\mathcal{H}(\Sigma)$ be the set of all hyperplanes contained in our Coxeter complex. For an alcove $c$ of \sg, let $\mathcal{H}(c)$ be the subset of $\mathcal{H}(\Sigma)$ which separates $c$ and the fixed identity alcove 1. Now $\mathcal{H}(c)=\mathcal{H}_{\phi}^+(c)\sqcup \mathcal{H}_{\phi}^-(c)$. 
\end{definition}


\begin{definition}
    Let $Ch(\Sigma)$ denote the set of all alcoves in \sg. The $\phi$-valuation map is the map $\textnormal{v}_\phi:\textnormal{Ch}(\Sigma)\longrightarrow\mathbb{Z}$, with
    \[c\mapsto \textnormal{v}_\phi(c):= |\mathcal{H}_{\phi}^+(c)|-|\mathcal{H}_{\phi}^-(c)|.\]
\end{definition}

\begin{definition}
    Let $p_\phi : \textnormal{Ch}(\Sigma)\times \mathcal{H}\longrightarrow \{0,1\}$ be the function 
    \[p_\phi(c,H):= \begin{cases}
        1 & \textnormal{if $c$ is on a $\phi$-positive side of $H$,}\\
        0 & \textnormal{otherwise.}
    \end{cases}\]
\end{definition}

\begin{lemma}
    \[\textnormal{v}_\phi(c)=\sum_{H\in\mathcal{H}(\Sigma)}(p_{\phi}(c,H)-p_\phi(1,H)).\]
\end{lemma}

\begin{proof}
    
\end{proof}

\begin{lemma}
    \[l(x)\geq\textnormal{v}_\phi(c_x).\]
\end{lemma}

\begin{proof}
    
\end{proof}

\begin{definition}
    We call an alcove $c$ \ix{dominant} with respect to $\phi$ if $\textnormal{v}_\phi(c)=l(c).$
\end{definition}

\begin{lemma}
    \[l(x)=\max_{a\in W_0}\textnormal{v}_{\tilde{\phi}_a}(c_x).\]
\end{lemma}

\begin{proof}
    
\end{proof}

\begin{lemma}
    Let $\phi\in$Dir$(W)$, $r\in W$ be a reflection across the hyperplane $H_r$ and $x\in W$. Then v$_\phi(x)>$v$_\phi(rx)$ if and only if $x$ lies in the $\phi$-positive side of $H_r$. 
\end{lemma}

\begin{proof}
    
\end{proof}















































































































































\bibliographystyle{plain}
\bibliography{references}



\end{document}