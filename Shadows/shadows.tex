\documentclass[12pt]{article}
\usepackage{amsmath}
\usepackage{amsthm}
\usepackage{graphicx}
\usepackage{amsfonts}
\usepackage{xcolor}
\begin{document}
\newtheorem{theorem}{Theorem}
\theoremstyle{definition}
\newtheorem{definition}{Definition}
\newtheorem{example}{Example}
\newtheorem{lemma}{Lemma}
\newtheorem{corollary}{Corollary}
\newcommand{\uw}{\mathcal{U}(W,X)}
\newcommand{\W}{$(W,S)$}
\newcommand{\ix}{\textit}
\newcommand{\tr}{\textcolor{red}}
\newcommand{\sg}{$\Sigma$}
\tableofcontents 



\textcolor{red}{Red notes for Megan}

\textcolor{blue}{Blue notes for Yusra}

Figures are not my own.

\section{Coxeter complexes}
\begin{definition}
    Let $(W,S)$ be a Coxeter system. Let $S'\subset S$. We define the \textit{standard parabolic subgroup} $W_{s'}$ of $W$ to be the subgroup generated by the subset $S'$. Then $(W_{s'},S')$ is also a Coxeter group. 
\end{definition}

We can now define an abstract simplicial complex $\Sigma$ by taking all the left cosets $xW_{S'}$ of all the standard parabolic subgroups, and defining a partial order on this set by reverse inclusion. 

\textcolor{red}{Why do we choose to order by reverse inclusion?}

The vertex set of this simplicial complex corresponds to cosets of the maximal parabolic subgroups. These maximal parabolic subgroups are formed by taking a subset of $S$ with one element removed. So these maximal parabolic subgroups are in bijection with the elements of $S$. 


\begin{definition}
    The maximal simplicies in the simplicial complex are called \textit{alcoves}, and the codimension-one faces are called \textit{panels}.  
\end{definition}

We notes that there is a correspondence between between panels and vertices. This is because a vertex corresponds to $xW_{S\backslash \{s\}}$, whilst a panel corresponds to an element if the form $xW_{S\backslash \{s\}}$. 

\begin{definition}
    If a panel $p$ corresonds to the element $xW_{S\backslash \{s\}}$, we say that $p$ has \ix{type} $s$, and write $\tau(p)=s$. 
\end{definition}

The we can consider $W$ acting on this simplicial complex.
We want to consider the set of elements of $W$ which exactly fix a hyperplane in the simplicial complex. This subset is
\[R:=\bigcup_{x\in W}xSx^{-1},\]
and the elements of this set are called reflections. Given an element $r\in R$, we denoted the hyperplane it fixes by $H_r$. Then the hyperplane $H_r$ \ix{separates} two alcoves if they are contained in different half-spaces defined by $H_r$. 

Now let us consider a Euclidean \textcolor{red}{(What actually is a Euclidean Coxeter group?)} Coxeter system of type $\tilde{X}$. This group can be split into a semi-direct product of a spherical Weyl group $W_0$ and a translation group $T$ which acts on $\Sigma$.
\begin{definition}
    A vertex of $\Sigma$ whose stabiliser in $W$ is isomorphic to $W_0$ is called a \ix{special vertex}.
\end{definition}

Now when we have an irreducible Euclidean Coxeter system, $\Sigma$ can be geometrically realised as a tiling of the Euclidean $n$-space, where $n=|S|-1$. Now here, the group $T$ is isomorphic to $\mathbb{Z}^n$. This correpsonds to the coroot lattice. 

Now let us consider this geometric realisation of $\Sigma$, which we also call $\Sigma$. Then we fix a special vertex 0, which we call the \ix{origin} of $\Sigma$. We want to consider the set $\mathcal{H}_v$ of all hyperplanes through a special vertex $v$ which is in the orbit of 0 under $T$. 

\begin{definition}
    The \ix{Weyl chambers} are the closures of the connected components of $\Sigma\backslash \bigcup_{H\in\mathcal{H}}H$.
\end{definition}

Now the set of equivalence classes of parallel rays in $\Sigma$ form what we call the \ix{boundary sphere}, denoted by $\partial\Sigma$. This sphere inherits a tiling from the oirginal tiling of the Euclidean plane. To do this, we take, as the hyperplanes, the parallel classes of hyperplanes in $\Sigma$. 





!!!!!!


\section{Orientations}

For this section, let \W be any Coxeter system, and $\Sigma$ be its associated Coxeter complex.

\begin{definition}
    An \ix{orientation} $\phi$ of \sg is a map from the set of pairs $(p,c)$, where $p$ is a panel and $c$ is an alcove containing $p$, to the set $\{+1,-1\}$. If $\phi (p,c)=+1$, then we say that $c$ is on the $\phi$-\ix{positive side}, otherwise we say that $c$ is on the $\phi$-\ix{negative side}. 
\end{definition}


\begin{example}
    The trivial positive orientation is the map which sends all pairs to $+1$. Similarly, the trivial negative orientation is the map which sends all pairs to $-1$. 
\end{example}

Often, we do not want to have orientations which locally behave like trivial orientations. Hence, we define the following concept:

\begin{definition}
    Given an orientation $\phi$ of \sg, we have
    \begin{enumerate}
        \item $\phi$ is \ix{locally non-negative} if, for each panel, there is at least one alcove which is on the $\phi$-positive side.
        \item $\phi$ is \ix{locally non-trivial} if, for every panel, there is exactly one alcove which is on the $\phi$-positive side.
    \end{enumerate}
\end{definition}


There is a natural action of $W$ on the set of all possible orientations of \sg, induced by the action of $W$ on on the alocves and panels. It is defined as 
\[(x\cdot\phi)(p,c):=(x^{-1}p,x^{-1}c)\]

\begin{definition}
    Given an orientation $\phi$ of \sg, we say that $\phi$ is \ix{wall consistent} if, given any wall $H$, all pairs $c,d$ of alcoves which lie in the same halfspace of $H$, with panels $p$ and $q$ respectively, we have that $\phi(p,c)=\phi(q,d)$. If our orientation is wall consistent, we can then define the \ix{positive side} $H^{\epsilon}$ of $H$ as the half-space such that all alcoves $c$ in $H^{\epsilon}$ have $\phi(p,c)=+1$ for all panels of $c$. Then the \ix{negative side} is defined similarly.
\end{definition}

We want to look at several natural ways to orient a Coxeter complex. First, we will look at an orientation which is derived from either a choice of alcove, or a choice of panel. This orientation works for any Coxeter group.

\begin{definition}
    Choose a fixed alcove $c$ in \sg. Now given any alcove $d$, and panel $p$, we define their orientation as $\phi(p,d)=+1$ if and only if $c$ and $d$ lie in the same side of the wall which is spanned by $p$. We call this orientation the \ix{alcove orientation towards c}.
\end{definition}

\begin{definition}
    Choose a fixed simplex $b$ in \sg. Now given any alcove $c$, and panel $p$ in $c$, we define their orientation as $\phi(q,c)=+1$ if and only if either $c$ and $b$ lie in the same side of the wall $H$ containing $p$, or if $b$ lies inside $H$. We call this orientation the \ix{simplex orientation towards b}.
\end{definition}
\begin{example}
    Here we see two simplex orientations of an $A_2$ Coxeter complex. In this complex, the alcoves are edges, and the panels are vertices.
\end{example}
\includegraphics[scale=0.6]{Screenshot 2023-02-03 102201.png}\\
\tr{How are we defining these hyperplanes in this case?}
\begin{lemma}
    Consider a Coxeter group \W with Coxeter complex \sg. We have the following:
    \begin{enumerate}
        \item If $\phi$ is a simplex orientation of \sg, then $\phi$ is wall consistent and locally non-negative.
        \item If $\phi$ is an alcove orientation of \sg, then $\phi$ is wall consistent and locally non-trivial.
    \end{enumerate}
\end{lemma}

\subsection{The affine case}

Now we want to consider when our Coxeter complex \sg is affine. To define an orientation on \sg, we choose a chamber at infinity.

If $\phi$ is a wall consistent orientation, then, given two chambers $c,d$ which share a common panel $p$, $c$ and $d$ are given the same orientation if they lie in the same half-space of the hyperplane spanned by $p$. This amounts to picking a positive side of the hyperplane.

However, we did not have to pick these positive sides in any consistent way. 

\begin{definition}
    Let $\phi$ be a wall consistent orientation of an affine Coxeter complex. We say that $\phi$ is \ix{periodic} if, given two parallel hyperplanes $H_1,H_2$ and corresponding half-spaces $H_1^{\epsilon},H_2^{\epsilon}$, if $H_1^{\epsilon}\subset H_2^{\epsilon}$, then $H_1^{\epsilon}$ is positive if and only if $H_2^{\epsilon}$ is positive. 
\end{definition}

\begin{example}
    If $\phi$ is a trivial orientation on an affine Coxeter complex, then $\phi$ is periodic. 
\end{example}

\begin{example}
    Simplex orientations are not periodic, as, for every set of parallel hyperplanes, we can find pairs ???
\end{example}

If $\phi$ is a periodic orientation, then we have a natural orientation which is induced on the boundary. 


?????


\section{Folded galleries}
\subsection{Definitions}
\begin{definition}
    Given a Coxeter complex \sg, a \ix{combinatorial gallery} is a sequence
    \[\gamma = (c_0,p_1,c_1,p_2,...,p_n,c_n),\]
    where the $c_i$ are alcoves and the $p_i$ are panels of \sg, such that $p_i$ is contained in $c_i$ and $c_{i-1}$ for all $i-1,...,n$. The length of a combinatorial gallery $\gamma$ is $n+1$. Then $\gamma$ is \ix{minimal} if there does not exist a shorter gallery starting at $c_0$ and ending at $c_n$. 
\end{definition}

So a gallery is a path between $c_0$ and $c_n$ through alcoves, such that adjacent alcoves in the path share a commmon panel. 

\begin{definition}
    Given a gallery $\gamma$ of \sg, we say that $\gamma$ is \ix{folded (or stammering)} if, within $\gamma$, we can find an index $i$ such that $c_i=c_{i-1}$. Then we say that $\gamma$ has a \ix{fold} at panel $p_i$. Otherwise, we say that $\gamma$ is \ix{unfolded (or non-stammering)}.  
\end{definition}

To represent a gallery, we draw a path which passes through every chamber and panel in the gallery of the Euclidean representation of our Coxeter complex. We draw an arrow towards the sink of our gallery. 

\includegraphics[scale=0.6]{Screenshot 2023-02-03 111653.png}\\

\begin{definition}
    Given a gallery $\gamma$ in \sg, and an orientation $\phi$, we say that $\gamma$ is \ix{positively folded} with respect to $\phi$ if, whenever $\gamma$ is folded at position $i$, $\phi(p_i,c_i)=+1$.  We can similarly define \ix{negatively folded}.
\end{definition}

This means that $\gamma$ is positively folded at $c_i=c_{i-1}$ if the repeated alcove $c_i$ lies on the positivelydefined side of $p_i$. 

\subsection{Galleries and Words}

\begin{definition}
    Consider a gallery $\gamma = (c_0,p_1,c_1,...,p_n,c_n)$. Let panel $p_i$ of $\gamma$ have type $s_{j_i}\in S$. We define its \ix{type} $\tau(\gamma)$ as the word 
    \[\tau(\gamma):=s_{j_1}...s_{j_n}.\]
    We denote by $\Gamma_{\phi}^+(w)$ the set of all $\phi$-positively folded galleries which have type $w$. 
\end{definition}

\begin{definition}
    The \ix{decorated type} $\hat\tau(\gamma)$ of a gallery $\gamma = (c_0,p_1,c_1,...,p_n,c_n)$ is the decorated word
    \[\hat\tau(\gamma):= s_{j_1}...\hat{s_{j_i}}...s_{j_n},\]
    where we place a hat on the elements $s_{j_i}$ of the word which correspond to a fold $c_{i-1}=c_i$ of the gallery. We denote by $\Gamma_{\phi}^+(\hat{w})$ the set of all $\phi$-positively folded galleries which have decorated type $\hat{w}$.
\end{definition}

\begin{lemma}
    Let $c_0$ be a chosen, fixed, alcove in our Coxeter complex \sg. 
    \begin{enumerate}
        \item There is a bijection between words in $S$ and unfolded galleries starting at $c_0$.
        \item There is a bijection between decorated words in $S$ and gallleries starting at $c_0$. 
    \end{enumerate}
\end{lemma}

\begin{lemma}
    Let $\gamma$ be a gallery. Then
    \begin{enumerate}
        \item $F(\gamma)=\emptyset$ of and only if $\tau(\gamma)=\hat{\tau}(\gamma).$
        \item $\gamma$ is minimal if and only if $F(\gamma)=\emptyset$ and $\tau(\gamma)$ is reduced
    \end{enumerate}
    \tr{What is the function F? I can't find the definition in this paper. Is it the set of all repeated chambers?}
\end{lemma}

We want to be able to characterise the last alcove in a gallery. We do this by constructing another gallary which removes any folds from our original gallery. This leads to an unfolded gallery which has shorter length than the original gallery.

\begin{definition}
    Consider a gallery $\gamma = (c_0,p_1,c_1,...,p_n,c_n)$ in \sg. We create a new gallery, called the \ix{footprint} \ix{ft}$(\gamma)$ \ix{of} $\gamma$, by deleting all pairs $p_i,c_i$ such that the letter $s_i$ has a hat in $\hat{\tau}(\gamma)$. 

\end{definition}

\includegraphics[scale=0.6]{Screenshot 2023-02-03 133522.png}\\

\begin{lemma}
    We can calculate the final alcove of a gallery as the element $c_n=c_0\cdot w$, where $w=\tau(\text{ft}(\gamma))$.
\end{lemma}

\subsection{Modification}

As $W$ has a natural left action on \sg, and so $W$ also acts on the set of galleries in \sg. For instance, $x\in W$ sends $\gamma = (c_0,p_1,c_1,...,p_n,c_n)$ to the gallery $\gamma = (xc_0,xp_1,xc_1,...,xp_n,xc_n)$. 

\begin{lemma}
    
\end{lemma}


\begin{definition}
    Consider a gallery $\gamma = (c_0,p_1,c_1,...,p_n,c_n)$. Let $H_i$ be the hyperplane containing the panel $p_i$, and let $r_i$ be the reflection across $H_i$. For $i=1,...,n$, let
    \[\gamma^i:=(c_o,p_1,...,p_i,r_ic_i,r_ip_{i+1},r_ic_{i+1},...,r_ip_n,r_ic_n).\]
    If $\gamma$ was folded at panel $p_i$, we call $\gamma^i$ a \ix{unfolding of }$\gamma$ at $p_i$. Otherwise, we call it a \ix{folding}.
\end{definition}

\begin{lemma}
    For all $i=1,...,n$, $\tau(\gamma)=\tau(\gamma^i)$. So folding and unfolding does not change the gallery type. Also, $(\gamma^i)^i=\gamma$.
\end{lemma}

\begin{lemma}
    For all $i,j=1,...,n$, $(\gamma^i)^j=(\gamma^j)^i.$
\end{lemma}

Because of this property, we are able to define a \ix{multifolding} with respect to a subset $I$ of $\{1,...,n\}$ as the (un-)foldings $\gamma^I$. Now multifoldng does not affect the type. Then the set of folds of $\gamma^I$ will be the symmetric difference of the folds of $\gamma$ and $I$. In particular, if $I$ and $J$ are subsets of $\{1,...,n\}$, $(\gamma^I)^J=\gamma^{I\Delta J}$. 

\begin{corollary}
    Given any gallery $\gamma$, there is a subset $I\subset \{1,...,n\}$ such that $\gamma^I$ is unfolded, and $\gamma$ and $\gamma^I$ have the same type.
\end{corollary}


\includegraphics[scale=0.6]{Screenshot 2023-02-03 153412.png}\\













\end{document}