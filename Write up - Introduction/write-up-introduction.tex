\documentclass[11pt]{article}
\usepackage[a4paper, total={6in, 8in}]{geometry}
\usepackage{amsmath}
\usepackage{amsthm}
\usepackage{graphicx}
\usepackage{amsfonts}
\usepackage{xcolor}
\begin{document}

\newtheorem{theorem}{Theorem}
\numberwithin{theorem}{section}
\theoremstyle{definition}
\newtheorem{definition}{Definition}
\newtheorem{proposition}{Proposition}
\newtheorem{example}{Example}
\newtheorem{lemma}{Lemma}
\newtheorem{corollary}{Corollary}
\numberwithin{definition}{section}
\numberwithin{proposition}{section}
\numberwithin{example}{section}
\numberwithin{lemma}{section}
\numberwithin{corollary}{section}
\newcommand{\uw}{\mathcal{U}(W,X)}
\newcommand{\W}{$(W,S)$}
\newcommand{\ix}{\textit}
\newcommand{\tr}{\textcolor{red}}
\newcommand{\sg}{$\Sigma$}


%\title{Buildings}
%\maketitle


\section{Introduction}

In this project, we will be looking at the geometric objects of buildings. Buildings can be defined in many ways. For instance, we can define buildings using a system of chambers, or Coxeter complexes. We can look at buildings as geometric objects, combinatorial objects, or as representations of groups. These viewpoints give us unique information of the building and group. We will look at galleries in buildings. These are walks around the chambers of a building. We then look at foldings of these galleries, with respect to orientations of our building. An important combinatorial question of foldings is which alcoves of the building can be reached by folding a certain gallery. We call this set the \ix{shadow} of a gallery. We shall see some progress in answering the question of calculating the shadow, and we will discuss tools which could be used to improve these answers. 






















































\bibliographystyle{plain}
\bibliography{references}

\end{document}