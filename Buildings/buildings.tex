\documentclass[12pt]{article}
\usepackage{amsmath}
\begin{document}
\newtheorem{definition}{Definition}
\newtheorem{theorem}{Theorem}
\newtheorem{example}{Example}

\section{Buildings}
\begin{definition}
  A Polyhedral complex is a certain finite dimensional CW-complex. Each n-cell of the polyhedral complex is   
\end{definition}

\begin{definition}
    Suppose P is a simple convex polytope in $X^n$. Let $F_i$ be the codimension-one faces of P. Suppose that, for any two faces Fi and Fj, if their intersection is non-empty, then the dihedral angle between the faces is pi/mij, for some mij in ${2,3,4,...}$. Now set $mii=1, mij=inf$ if Fi, Fj empty intersection. Let $s_i$ be the reflection of $X^n$ across Fi, and let W be the group generated by the set of si's. Then W is the Coxeter group with generators $s_i$, and Coxeter matrix $(mij)$. Furthermore, W is a discrete subgroup of Isom($X^n$), P is a strict fundamental domain for the W action, and P tiles $X^n$. 
\end{definition}   
    
\begin{definition}
    Let $(W,S)$ be a Coxeter group generated by a simple convex polytope P. A building of type $(W,S)$ is a polyhedral complex, which is a union of subcomplexs, called apartments. An apartment is isometric to the tiling of $X^n$ derived from P, and each copy of P in the tiling is called a chamber. Now the apartments and chambers must satisfy
    \begin{enumerate}
    \item Given any two chambers, there exists an aprtment containing both of them.
    \item Given any two apartments A and B, there exists an isometry from A to B which fixes $A \cap B$ pointwise. 
    \end{enumerate}
\end{definition}

\begin{example}
    Let us consider a single copy of $X^n$. We can tile this copy by P, and we get a thin building. This means that we only have one apartment. Clearly this satisfies the first condition - any two chambers immediately lie in the only apartment. 

    Now let us look at the second condition. If the two chambers have no intersection, then, as each chamber is a copy of P, they are clearly isometric, and we are done. Now if the two chambers have a non-empty intersection, we have two cases:
    \begin{enumerate}
        \item If they share a common edge, then reflection along this edge gives us our isometry.
        \item If they only share a common point
    \end{enumerate}
\end{example}

\begin{example}
    Now we can consider a spherical building. Take the Coxeter group
    \[W=\langle s_1,s_2 | s_i^2=1, (s_1s_2)^2=1\rangle.
    \]
    This Coxeter group is isomorphic to $D_4$. 
\end{example}
\end{document}