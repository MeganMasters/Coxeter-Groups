\documentclass[11pt]{article}
\usepackage[a4paper, total={6in, 8in}]{geometry}
\usepackage{amsmath}
\usepackage{amsthm}
\usepackage{graphicx}
\usepackage{amsfonts}
\usepackage{xcolor}
\usepackage{enumerate}
\begin{document}

\newtheorem{theorem}{Theorem}
\numberwithin{theorem}{section}
\theoremstyle{definition}
\newtheorem{definition}{Definition}
\newtheorem{proposition}{Proposition}
\newtheorem{example}{Example}
\newtheorem{lemma}{Lemma}
\newtheorem{corollary}{Corollary}
\numberwithin{definition}{section}
\numberwithin{proposition}{section}
\numberwithin{example}{section}
\numberwithin{lemma}{section}
\numberwithin{corollary}{section}
\newcommand{\uw}{\mathcal{U}(W,X)}
\newcommand{\W}{$(W,S)$}
\newcommand{\ix}{\textit}
\newcommand{\tr}{\textcolor{red}}
\newcommand{\sg}{$\Sigma$}


%\title{Buildings}
%\maketitle


\section{Progress on the question}


\subsection{Statistics on positive folds}

We now restrict to looking at Weyl chamber orientations over affine Coxeter complexes. This means that we have a complex \sg, with a boundary $\partial$\sg, and that our orientations are induced by a boundary chamber orientation. Here, we can get a partial answer to our main question of calculating the shadow of a given gallery. To do this, we define a $\phi$-valuation map on our set of alcoves. We can then prove a recursive algorithm for calculating the shadow of a gallery.

First, given a gallery, we want to calculate the number of positive folds of this gallery that we can make. A proof of this proposition can be found in \cite{DEL}.

\begin{proposition}
    Consider the largest element $w_0$ in $W_0$. Given an $x\in W$, and a $\phi$-postive (multi)folding $\gamma$ of $\gamma_x$, we have
    \[l_R(xy^{-1})\leq |F(\gamma)|\leq l(w_0),\]
    where $y:=\tau($ft$(\gamma))$.
\end{proposition}


\begin{definition}
    Let $\mathcal{H}(\Sigma)$ be the set of all hyperplanes contained in our Coxeter complex. For an alcove $c$ of \sg, let $\mathcal{H}(c)$ be the subset of $\mathcal{H}(\Sigma)$ which separates $c$ and the fixed identity alcove 1. Now $\mathcal{H}(c)=\mathcal{H}_{\phi}^+(c)\sqcup \mathcal{H}_{\phi}^-(c)$. 
\end{definition}


\begin{definition}
    Let $Ch(\Sigma)$ denote the set of all alcoves in \sg. The $\phi$-valuation map is the map $\textnormal{v}_\phi:\textnormal{Ch}(\Sigma)\longrightarrow\mathbb{Z}$, with
    \[c\mapsto \textnormal{v}_\phi(c):= |\mathcal{H}_{\phi}^+(c)|-|\mathcal{H}_{\phi}^-(c)|.\]
\end{definition}

\begin{definition}
    Let $p_\phi : \textnormal{Ch}(\Sigma)\times \mathcal{H}\longrightarrow \{0,1\}$ be the function 
    \[p_\phi(c,H):= \begin{cases}
        1 & \textnormal{if $c$ is on a $\phi$-positive side of $H$,}\\
        0 & \textnormal{otherwise.}
    \end{cases}\]
\end{definition}

We now want to relate this function to our $\phi$-valuation map.

\begin{lemma}
    \[\textnormal{v}_\phi(c)=\sum_{H\in\mathcal{H}(\Sigma)}(p_{\phi}(c,H)-p_\phi(1,H)).\]
\end{lemma}

\begin{proof}
    We are assuming that our oritentation $\phi$ is a chamber orientation. So, in particular, this orientation is locally non-trivial. Therefore, every hyperplane $H$ has a positive and negative side. First consider when 1 and $c$ lie on the same side of $H$. Then $H$ is not an element of $\mathcal{H}(c)$. But in this case, $p_\phi(1,H)=p_\phi(c,H)$ and so this hyperplane does not contribute to the above sum. 
    Now consider when 1 and $c$ lie on opposite sides of $H$. In this case, $H\in \mathcal{H}(c)$. If $c$ lies on the positive side of $H$, then $H\in\mathcal{H}_{\phi}^+(c)$ and $p_\phi(c,H)=1$ and $p_\phi(1,H)=0$, and so $H$ contributes $+1$ to the sum above. Similarly, if $c$ lies on the negative side of $H$, then $H\in\mathcal{H}_{\phi}^-(c)$ and $p_\phi(c,H)=0$ and $p_\phi(1,H)=1$, and so $H$ contributes $-1$ to the sum above. Therefore, we are just counting the size of $\mathcal{H}_{\phi}^+(c)$ minus the size of $\mathcal{H}_{\phi}^-(c)$, which is exactly $\textnormal{v}_\phi(c)$. 
\end{proof}


The next lemma comes from the trivial observation that \[|\mathcal{H}_{\phi}^+(c)|+|\mathcal{H}_{\phi}^+(c)|\geq |\mathcal{H}_{\phi}^+(c)|-|\mathcal{H}_{\phi}^+(c)|.\]
\begin{lemma}
    \[l(x)\geq\textnormal{v}_\phi(c_x).\]
\end{lemma}

\begin{definition}
    We call an alcove $c$ \ix{dominant} with respect to $\phi$ if $\textnormal{v}_\phi(c)=l(c).$
\end{definition}

\begin{lemma}
    \[l(x)=\max_{a\in W_0}\textnormal{v}_{\tilde{\phi}_a}(c_x).\]
\end{lemma}

\begin{proof}
    
\end{proof}

\begin{lemma}
    Let $\phi\in$Dir$(W)$, $r\in W$ be a reflection across the hyperplane $H_r$ and $x\in W$. Then v$_\phi(x)>$v$_\phi(rx)$ if and only if $x$ lies in the $\phi$-positive side of $H_r$. 
\end{lemma}

\begin{proof}
    
\end{proof}

\subsection{Computation of regular shadows}

We now want to see how we can use this new valuation map to define a recursive definition of a shadow. To do this, we need the next important theorem. A proof of this theorem can be found in \cite[pp.142-143]{SHA}. 

Let Dir$(W)$ represent the set of chambers in the boundary complex $\partial\Sigma$. We call elements of Dir$(W)$ \ix{directions in W}. 

\begin{theorem}
    Let $\phi\in$Dir$(W)$, $x\in W$ and $s\in S$. Then
    \begin{enumerate}[(i)]
        \item If $s$ is in the right descent set $D_R(x)$ of $x$, then we have
        \[\textnormal{Sh}_\phi(x)=\textnormal{Sh}_\phi(xs)\cdot s \cup \{z\in \textnormal{Sh}_\phi(xs):\textnormal{v}_\phi(zs)<\textnormal{v}_\phi(z)\}.\]
        \item If $s$ is in the left descent set $D_R(x)$ of $x$, then we have
        \[\textnormal{Sh}_\phi(x)=\begin{cases}
            s\cdot \textnormal{Sh}_\phi(sx)\cup \textnormal{Sh}_\phi(sx) &if\textnormal{ v}_\phi(s)<0,\\
            s\cdot \textnormal{Sh}_\phi(sx) &if \textnormal{ v}_\phi(s)>0.\\
        \end{cases}\]
    \end{enumerate}
\end{theorem}


%\begin{definition}
   % Given $x\in W$, $a\in W_0$ and $\phi\in$ Dir$(W)$, the \ix{partial shadow in local direction a} is the set 
   % \[\textnormal{Sh}_\phi^a(x):=\{y\in \textnormal{Sh}_\phi(x)|\bar{y}=a\},\]
   % where $\bar{y}$ is the image of $y$ under the natural projection to the spherical Weyl group $W_0$. 
%\end{definition}

Now we can use this theorem to show that the next two lemmas both give us recursive defintions for the shadow of a gallery. 

\begin{lemma} (Algorithm L)
    Let $\phi\in\textnormal{Dir}(W)$ and $x\in W$. Let $w=(s_1,...,s_n)$ be a reduced word for $x$. Let $A_0=\{1\}$ and let
    \[A_i:=A_{i-1}\cdot s_i\cup \{z\in A_{i-1}|v_\phi(zs)<v_\phi(z)\}.\]
    Then $A_n=\textnormal{Sh}_\phi(x)$. 
\end{lemma}

\begin{proof}
    Using the theorem above, we can show by induction that $A_i=$Sh$_\phi(s_1...s_i)$ for $i=0,...,n$. Firstly, for $i=0$ it is trivial, as Sh$(1)=\{1\}$. Then assume that $A_i=$Sh$_\phi(s_1...s_i)$ for $i<j$. By part (i) of the theorem,\[\begin{aligned}
    \textnormal{Sh}(s_1...s_j)& = \textnormal{Sh}(s_1...s_js_j)\cdot s_j \cup \{z\in \textnormal{Sh}(s_1...s_js_j): \textnormal{v}_\phi(zs)<\textnormal{v}_\phi(z)\}\\
        & = \textnormal{Sh}(s_1...s_{j-1})\cdot s_j \cup \{z\in \textnormal{Sh}(s_1...s_{j-1}): \textnormal{v}_\phi(zs)<\textnormal{v}_\phi(z)\}\\
        & = A_{j-1}\cdot s_j \cup \{z\in A_{j-1}: \textnormal{v}_\phi(zs)<\textnormal{v}_\phi(z)\}\\
        & = A_j.
    \end{aligned}\]
\end{proof}

\begin{lemma} (Algorithm R)
    Let $\phi\in$ Dir$(W)$ and $x\in W$, with $(s_n,...,s_1)$ a reduced expression for $x$. Let $B_0^\phi:=\{1\}$ and define
    \[B_i^\phi = \begin{cases}
        s_iB_{i-1}^{s_i\phi}\cup B_{i-1}^\phi & \textnormal{if v}_\phi(s_i)<0,\\
        s_iB_{i-1}^{s_i\phi} & \textnormal{if v}_\phi(s_i)>0.
    \end{cases}\]
    Then $B^\phi_n=$Sh$_\phi(x)$ for all $\phi\in$ Dir$(W)$. 
\end{lemma}

\begin{proof}
    Again, we can use the theorem above to prove by induction that $B^\phi_i=$Sh$_\phi(s_i...s_1)$ for all $i=0,...,n$. For $i=0$ it is trivial as Sh$_\phi(1)=\{1\}$. Now assume that $B^\phi_i=$Sh$_\phi(s_i...s_1)$ for all $i<j$. By part (ii) of the theorem, if $v(s_j)<0$, then
    \[\begin{aligned}
        Sh_\phi(s_j...s_1)&=s_j\cdot \textnormal{Sh}_{s_j\phi}(s_js_j...s_1)\cup \textnormal{Sh}_\phi(s_js_j...s_1)\\
                        &=  s_j\cdot \textnormal{Sh}_{s_j\phi}(s_{j-1}...s_1)\cup \textnormal{Sh}_\phi(s_{j-1}...s_1)\\
                        &= s_j\cdot B^{s_j\phi}_{j-1}\cup B^{\phi}_{j-1}\\
                        &= B^\phi_j. 
    \end{aligned}\]
    Similarly, if $v(s_j)<0$, then
    \[\begin{aligned}
        Sh_\phi(s_j...s_1)& = s_j\cdot \textnormal{Sh}_{s_j\phi}(s_js_j...s_1)\\
                        & = s_j\cdot \textnormal{Sh}_{s_j\phi}(s_{j-1}...s_1)\\
                        & = s_j \cdot B^{s_j\phi}_{j-1}\\
                        & = B^{s_j\phi}_{j}. 
    \end{aligned}\]
\end{proof}












































































































































\bibliographystyle{plain}
\bibliography{references}



\end{document}