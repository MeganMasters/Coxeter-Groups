\documentclass[12pt]{article}
\usepackage{amsmath}
\usepackage{amsthm}
\usepackage{graphicx}
\usepackage{amsfonts}
\usepackage{xcolor}
\begin{document}
\newtheorem{theorem}{Theorem}
\theoremstyle{definition}
\newtheorem{definition}{Definition}
\newtheorem{example}{Example}
\newtheorem{lemma}{Lemma}
\newcommand{\uw}{\mathcal{U}(W,X)}
\newcommand{\W}{$(W,S)$}
\newcommand{\ix}{\textit}
\newcommand{\tr}{\textcolor{red}}
\newcommand{\sg}{$\Sigma$}
\tableofcontents 



\textcolor{red}{Red notes for Megan}

\textcolor{blue}{Blue notes for Yusra}
\section{Coxeter complexes}
\begin{definition}
    Let $(W,S)$ be a Coxeter system. Let $S'\subset S$. We define the \textit{standard parabolic subgroup} $W_{s'}$ of $W$ to be the subgroup generated by the subset $S'$. Then $(W_{s'},S')$ is also a Coxeter group. 
\end{definition}

We can now define an abstract simplicial complex $\Sigma$ by taking all the left cosets $xW_{S'}$ of all the standard parabolic subgroups, and defining a partial order on this set by reverse inclusion. 

\textcolor{red}{Why do we choose to order by reverse inclusion?}

The vertex set of this simplicial complex corresponds to cosets of the maximal parabolic subgroups. These maximal parabolic subgroups are formed by taking a subset of $S$ with one element removed. So these maximal parabolic subgroups are in bijection with the elements of $S$. 


\begin{definition}
    The maximal simplicies in the simplicial complex are called \textit{alcoves}, and the codimension-one faces are called \textit{panels}.  
\end{definition}

We notes that there is a correspondence between between panels and vertices. This is because a vertex corresponds to $xW_{S\backslash \{s\}}$, whilst a panel corresponds to an element if the form $xW_{S\backslash \{s\}}$. 

\begin{definition}
    If a panel $p$ corresonds to the element $xW_{S\backslash \{s\}}$, we say that $p$ has \ix{type} $s$, and write $\tau(p)=s$. 
\end{definition}

The we can consider $W$ acting on this simplicial complex.
We want to consider the set of elements of $W$ which exactly fix a hyperplane in the simplicial complex. This subset is
\[R:=\bigcup_{x\in W}xSx^{-1},\]
and the elements of this set are called reflections. Given an element $r\in R$, we denoted the hyperplane it fixes by $H_r$. Then the hyperplane $H_r$ \ix{separates} two alcoves if they are contained in different half-spaces defined by $H_r$. 

Now let us consider a Euclidean \textcolor{red}{(What actually is a Euclidean Coxeter group?)} Coxeter system of type $\tilde{X}$. This group can be split into a semi-direct product of a spherical Weyl group $W_0$ and a translation group $T$ which acts on $\Sigma$.
\begin{definition}
    A vertex of $\Sigma$ whose stabiliser in $W$ is isomorphic to $W_0$ is called a \ix{special vertex}.
\end{definition}

Now when we have an irreducible Euclidean Coxeter system, $\Sigma$ can be geometrically realised as a tiling of the Euclidean $n$-space, where $n=|S|-1$. Now here, the group $T$ is isomorphic to $\mathbb{Z}^n$. This correpsonds to the coroot lattice. 

Now let us consider this geometric realisation of $\Sigma$, which we also call $\Sigma$. Then we fix a special vertex 0, which we call the \ix{origin} of $\Sigma$. We want to consider the set $\mathcal{H}_v$ of all hyperplanes through a special vertex $v$ which is in the orbit of 0 under $T$. 

\begin{definition}
    The \ix{Weyl chambers} are the closures of the connected components of $\Sigma\backslash \bigcup_{H\in\mathcal{H}}H$.
\end{definition}

Now the set of equivalence classes of parallel rays in $\Sigma$ form what we call the \ix{boundary sphere}, denoted by $\partial\Sigma$. This sphere inherits a tiling from the oirginal tiling of the Euclidean plane. To do this, we take, as the hyperplanes, the parallel classes of hyperplanes in $\Sigma$. 





!!!!!!


\section{Orientations}

For this section, let \W be any Coxeter system, and $\Sigma$ be its associated Coxeter complex.

\begin{definition}
    An \ix{orientation} $\phi$ of \sg is a map from the set of pairs $(p,c)$, where $p$ is a panel and $c$ is an alcove containing $p$, to the set $\{+1,-1\}$. If $\phi (p,c)=+1$, then we say that $c$ is on the $\phi$-\ix{positive side}, otherwise we say that $c$ is on the $\phi$-\ix{negative side}. 
\end{definition}


\begin{example}
    The trivial positive orientation is the map which sends all pairs to $+1$. Similarly, the trivial negative orientation is the map which sends all pairs to $-1$. 
\end{example}

Often, we do not want to have orientations which locally behave like trivial orientations. Hence, we define the following concept:

\begin{definition}
    Given an orientation $\phi$ of \sg, we have
    \begin{enumerate}
        \item $\phi$ is \ix{locally non-negative} if, for each panel, there is at least one alcove which is on the $\phi$-positive side.
        \item $\phi$ is \ix{locally non-trivial} if, for every
    \end{enumerate}
\end{definition}

































































\end{document}