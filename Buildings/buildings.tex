\documentclass[12pt]{article}
\usepackage{amsmath}
\usepackage{amsthm}
\usepackage{graphicx}
\usepackage{amsfonts}
\usepackage{xcolor}
\begin{document}
\newtheorem{theorem}{Theorem}
\theoremstyle{definition}
\newtheorem{definition}{Definition}
\newtheorem{example}{Example}
\newtheorem{lemma}{Lemma}
\newcommand{\uw}{\mathcal{U}(W,X)}
\tableofcontents
\section{Buildings}
\begin{definition}
  A Polyhedral complex is a certain finite dimensional CW-complex. Each n-cell of the polyhedral complex is   
\end{definition}

\begin{definition}
    Suppose P is a simple convex polytope in $X^n$. Let $F_i$ be the codimension-one faces of P. Suppose that, for any two faces Fi and Fj, if their intersection is non-empty, then the dihedral angle between the faces is pi/mij, for some mij in ${2,3,4,...}$. Now set $mii=1, mij=inf$ if Fi, Fj empty intersection. Let $s_i$ be the reflection of $X^n$ across Fi, and let W be the group generated by the set of si's. Then W is the Coxeter group with generators $s_i$, and Coxeter matrix $(mij)$. Furthermore, W is a discrete subgroup of Isom($X^n$), P is a strict fundamental domain for the W action, and P tiles $X^n$. 
\end{definition}   
    
\begin{definition}
    Let $(W,S)$ be a Coxeter group generated by a simple convex polytope P. A building of type $(W,S)$ is a polyhedral complex, which is a union of subcomplexs, called apartments. An apartment is isometric to the tiling of $X^n$ derived from P, and each copy of P in the tiling is called a chamber. Now the apartments and chambers must satisfy
    \begin{enumerate}
    \item Given any two chambers, there exists an aprtment containing both of them.
    \item Given any two apartments A and B, there exists an isometry from A to B which fixes $A \cap B$ pointwise. 
    \end{enumerate}
\end{definition}

\begin{example}
    Let us consider a single copy of $X^n$. We can tile this copy by P, and we get a thin building. This means that we only have one apartment. Clearly this satisfies the first condition - any two chambers immediately lie in the only apartment. 

    Now let us look at the second condition. If the two chambers have no intersection, then, as each chamber is a copy of P, they are clearly isometric, and we are done. Now if the two chambers have a non-empty intersection, we have two cases:
    \begin{enumerate}
        \item If they share a common edge, then reflection along this edge gives us our isometry.
        \item If they only share a common point
    \end{enumerate}
\end{example}

\begin{example}
    Now we can consider a spherical building. Take the Coxeter group
    \[W=\langle s_1,s_2 | s_i^2=1, (s_1s_2)^2=1\rangle.
    \]
    This Coxeter group is isomorphic to $D_4$. 
\end{example}

\section{Cayley graphs}
\begin{definition}
    The Cayley graph Cay(G,S) of a group $G$ with respect to a generating set $S$, $1\notin S$, is the graph $(V,E)$, with $V=G$, and directed edges
    \[E=\{(g,gs)|g\in G, s\in S\}.\]
    If $s\in S$ is an involution, we only put a single undirected edge between $g$ and $gs$, and label the edge $s$. 
\end{definition}

\begin{example}
    The Cayley graph of $D_6$, with generating set $\{s_1,s_2\}$ is 


    %\centering{\includegraphics[scale=0.4]{Screenshot 2023-01-03 at 12.40.32.png}}
\end{example}
\section{Reflection systems}
\begin{definition}
    Let $G$ be a group. A pre-reflection system for $G$ is a pair $(X,R)$. $X$ is a connected simplicial graph which is acted upon by $G$, and $R$ is a subset of $G$. This must satisfy
    \begin{enumerate}
        \item every element of $R$ is an involution;
        \item R is closed under conjugation;
        \item R generates G;
        \item given an edge of $X$, there is a unique element of $R$ which flips the edge; and
        \item for every element $r$ of $R$, there is at least one edge of $X$ which is flipped by $r$. 
    \end{enumerate}
\end{definition}

\begin{example}
    Let $(W,S)$ be any Coxeter system. Let $X$ be the Cayley graph of $(W,S)$, and let
    \[ R=\{wsw^{-1}|w\in W, s\in S\}. \]
    Then $(X,R)$ is a pre-reflection system.
\end{example}

\begin{definition}
    Consider a pre-reflection system $(X,R)$. For each element $r$ of $R$, the wall $H_r$ is the set of midpoints of all the edges flipped by $r$. 
\end{definition}

\begin{definition}
    Consider a pre-reflection system $(X,R)$. If, additionally, it satisfies
    \begin{enumerate}
        \setcounter{enumi}{5}
        \item for every element $r$ in $R$, $X\backslash H_r$ has exactly two components,
    \end{enumerate}
    then $(X,R)$ is a reflection system. 
\end{definition}


\begin{theorem}
    Suppose we have a group $W$, generated by a set $S$ of distinct involutions. Then the following are equivalent:
    \begin{enumerate}
        \item $(W,S)$ is a Coxeter system;
        \item $(X,R)$ is a reflection system, where $X=Cay(W,S)$ and \[R=\{wsw^{-1}|w\in W, s\in S\}.\]
        \item $(W,S)$ satisfies the Deletion Condition; and
        \item $(W,S)$ satisfies the Exchange Condition.
    \end{enumerate}
\end{theorem}


\section{Tits' solution to the word problem}

\begin{definition}
    Let $W$ be generated by a set $S$ of distinct involutions. Let $s,t\in S$, with $s\neq t$, and let $m_{st}$ be the order of $st$ in $W$. If $m_{st}$ is finite, consider a word in $S$ with the subword $sts...$ with $m_{st}$ letters. A braid move on the word replaces the subword $sts...$ with $tst...$, again with $m_{st}$ letters. 
\end{definition}

\begin{theorem}
    (Tits) Suppose we have a group $W$, generated by a set $S$ of distinct involutions. Also suppose that the Exchange Condition holds. Then
    \begin{enumerate}
        \item A word $s_1s_2...s_k$ is reduced iff we cannot shorten it by a sequence of 
        \begin{itemize}
            \item deleting an instance of $ss$ from the word, or
            \item applying a braid move to the word.
        \end{itemize}
        \item Two reduced words will represent the same group element iff a sequence of braid moves sends one to the other.
    \end{enumerate}
\end{theorem}


\section{Tits' Representation Theorem}
\begin{theorem}
    (Tits' Representation Theorem) Let $(W,S)$ be a Coxeter system. Then there is a map
    \[\rho : W \longrightarrow GL(N,\mathbb{R})\]
    which is a faithful representation, with $N=|S|$, such that
    \begin{enumerate}
        \item $\sigma_i = \rho (s_i)$ is a linear involution, whose fixed set is a hyperplane; and 
        \item If $s_i,s_j\in S$ are distinct, then $\sigma_i\sigma_j$ has order $m_{ij}$.  
    \end{enumerate}
\end{theorem}

\begin{definition}
    The representation $\rho$ above is called the Tits representation, or sometimes the standard geometric representation. 
\end{definition}

\subsection{Corollaries}

\bigbreak

\section{Construction of a geometric realisation}
\subsection{Simplicial complexes}
\begin{definition}
    Let $V$ be a, possibly infinite, set, called the vertex set. Let $X$ be a collection of finite subsets of $V$ such that 
    \begin{enumerate}
        \item $\{v\}\in X$ for all elements $v\in V$; and
        \item if $\Delta \in X$ and $\Delta'$ is a subset of $\Delta$, then $\Delta'$ is in $X$.
    \end{enumerate}
    Then $(V,X)$ is an abstract simplicial complex.
\end{definition}

\begin{definition}
    A simplex is any element of $X$. A simplex $\Delta$ has dimension 
    \[\dim \Delta = |\Delta| -1.\]
    A simplex of dimension $k$ is called a k-simplex. A 0-simplex is called a vertex, and a 1-simplex is called an edge, intuitively. The set of all simplices of dimension k is the k-skeleton $X^{(k)}.$ 
\end{definition}

\begin{lemma}
    The k-skeleton is also a simplicial complex. 
\end{lemma}

\begin{definition}
    The dimension of $X$ is
    \[\dim X = max\{\dim(\Delta)\\Delta \in X\}.\]
    If all the maximal elements of $X$ have the same dimension, then the simplicial complex is pure. 
\end{definition}
\begin{definition}
    The standard n-simplex $\Delta^n$ is the convex hull of the $(n+1)$ points $(1,0,...,0),...,(0,...,0,1)$ in $\mathbb{R}^{n+1}$.
\end{definition}
Given an n-simplex $\Delta$ in $X$, we can identify $\Delta$ with a copy of $\Delta^n$. 
\subsection{The basic construction}
\begin{definition}
    Let $X$ be a connected Hausdorff topological space. Let $(W,S)$ be a Coxeter system. Let $(X_s)_{s\in X}$ be a collection of non-empty, closed subspaces of $X$. Then $(X_s)_{s\in X}$ is a mirror structure on X over S, and $X_s$ is called the s-mirror. 
\end{definition}

\begin{example}
    Consider the cone with $|S|$ vertices. This is the graph with a node in the centre, and a branch for each element in $|S|$. Label the vertices $\{\sigma_s|s\in S\}$. Then we can set $X_s=\sigma_s$. This means that we take, for each element of $S$, the closed set of a single point as the $s$-mirror.
    %\center{\includegraphics[scale=0.4]{Screenshot 2023-01-09 at 12.31.42.png}}
\end{example}

\begin{example}
    Consider the $n$-simplex, with $n=|S|-1$. Let $\{\Sigma_s|s\in S\}$ be the codimension-one faces. Now let $X_s=\Sigma_s$. This means that we take, for every element of $S$, a codimension-one element of the simplex as the $s$-mirror.
    %\center{\includegraphics[scale=0.6]{Screenshot 2023-01-09 at 16.14.41.png}}
\end{example}

\begin{definition}
    For each $x\in X$, define the set
    \[S(x):=\{s\in S|x\in X_s\}.\]
\end{definition}

\begin{example}
    From example 5, we have
    \[S(x)=\begin{cases}\emptyset, & \text{if } x\notin \{\sigma_s|s\in S\},\\
        \{s\}, & \text{if } x=\sigma_s.
    \end{cases}\]
\end{example}

\begin{example}
    From example 6, we have
    \[\]
\end{example}

We now want to define an equivalence relation on $W\times X$. 
\begin{definition}
    $(w,x)$ is equivalent to $(w'x')$, i.e $(w,x)\sim (w'x')$, if and only if $x=x'$ and $w^{-1}w'\in W_{S(x)}.$  
\end{definition}

Now we want to equip our group $W$ with the discrete topology, and then $W\times X$ with the product topology. Then we define
\[\mathcal{U}(W,X)=W\times X/\sim.\]
Now we will denote by $[w,x]$ the equivalence class of $(w,x)$, and we will write $wX$ for the image of $\{w\}\times X$ in $\uw$. Now this must be well-defined, as $x\mapsto [w,x]$ is an embedding. We call each $wX$ a chamber. 

\begin{example}
    Let $W$ be the $(3,3,3)$-triangle group, i.e
    \[W=\langle s,t,u|s^2=t^2=u^2=1,(st)^3=(tu)^3=(us)^3=1\rangle.\]
    Now we let our topological space be $X=Cone\{\sigma_s,\sigma_t,\sigma_u\}$. So we have
    \[S(x)=\begin{cases}
        \emptyset, & \text{if } x\notin \{\sigma_s,\sigma_t,\sigma_u\},\\
        \{s\}, \{t\}, \{u\} & \text{if } x=\sigma_s,\sigma_t,\sigma_u \text{ resp.}
    \end{cases}\]
    So $W_{S(x)}$ is one of $1, \{1,s\},\{1,t\},$ or $\{1,u\}$. 
\end{example}

\begin{definition}
    Let $(W,S)$ be a Coxeter system. Let $X$ be a simplex with codimension-one faces $\{\Delta_s|s\in S\}$. Then $\uw$ is called the Coxeter complex.
\end{definition}

\subsection{Properties of $\uw$}
\begin{lemma}
    As a topological space ,$\uw$ is connected.
\end{lemma}

\begin{definition}
    We define  $\uw$ as locally finite if, given $[w,x]\in\uw$, we can find an open neighbourhood of $[w,x]$ which meets only a finite number of chambers.
\end{definition}

\begin{lemma}
    The following are equivalent:
    \begin{enumerate}
        \item $\uw$ is locally finite;
        \item given any $x\in X$, $W_{S(x)}$ is finite;
        \item Given any $T\subset S$ such that its special subgroup $W_T$ is infinite, we have $\cap_{x\in T}X_t=\emptyset.$
    \end{enumerate}
\end{lemma}

\begin{example}
    Let $W=\langle s,t,u|s^2=t^2=u^2=1,(st)^3=(tu)^3=(us)^3=1\rangle$. The Coxeter complex of $W$ is not locally finite. 
\end{example}


\subsection{Action of W on $\uw$}

We note that $W$ acts naturally on $W\times X$ by $w'\cdot(w,x)=(w'w,x)$. 

\begin{lemma}
    $W$ acts on $\uw$ by $w'\cdot[w,x]=[w'w,x].$
\end{lemma}

We note that this action also induces an action on the set of chambers. On the set of chambers, this action is transitive, and is free if there is a point $x\in X$ which is not contained in any mirror. 

Now for the point $[w,x]\in\uw$, its stabiliser is $wW_{S(x)}w^{-1}$. 

\begin{definition}
    Let $G$ be a discrete group, and let $Y$ be a Hausdorff space. An action by homeomorphisms of $G$ on $Y$ is properly discontinuous if 
    \begin{enumerate}
        \item $Y/G$ is Hausdorff;
        \item for any $y\in Y$, $G_y=$stab$_G(Y)$ is finite;
        \item for any $y\in Y$, we can find an open nieghbourhood $U_y$ of $y$ such that $U_y$ is stabilised by $G_y$, and $gU_y\cap U_y=\emptyset$ for all $g\in G\backslash G_y$. 
    \end{enumerate}
\end{definition}

\begin{lemma}
    THe action of $W$ on $\uw$ is properly discontinuous if and only if $W_{S(x)}$ is finite for all $x\in X$. 
\end{lemma}

\subsection{Universal property}

We now claim that $\uw$ satisfies the following universal property.

\begin{theorem}
    (Vinberg) Let $(W,S)$ be any Coxeter system. Let $W$ act by homeomorphisms on a connected Hausdorff space $Y$. Assume that for any $s\in S$, the fixed point set $Y^s$ is non-empty. Assume that $X$ is a connected Hausdorff space, and has a mirror structure $(X_S)_{s\in S}$. Let $f;X\longrightarrow Y$ be a continuous map with $f(X_s)\subset Y^s$ for all $s\in S$. Then there is a unique extension of f to a $W$-equivariant map $\tilde{f}:\uw \longrightarrow Y$. This map is given by $\tilde{f}([w,x])=w\cdot f(x)$. 
\end{theorem}

\section{Geometric Reflection groups and the Davis Complex}
\begin{theorem}
    Let $X$ be the simple convex polytope, $(W,S)$ the Coxeter group etc. Let $\bar{s}_i$ be the reflection in $F_i$, and let $\bar{W}$ be the group generated by the reflections. Then 
    \begin{enumerate}
        \item the map $\phi: W\longrightarrow\bar{W}$, induced by $s_i\mapsto \bar{s}_i$, is an isomorphism;
        \item the induced map $\mathcal{U}(W,P^n)$ is a homeomorphism;
        \item the Coxeter group $W$ acts properly discontinuously on $\mathbb{X}^n$, with strict fundamental domain $P^n$. Therefore, $W$ is a discrete subgroup of Isom($\mathbb{X}^n$) and $\mathbb{X}^n$ is tiled by copies of $P^n$. 
    \end{enumerate}

\end{theorem}

To prove this theorem, the idea is that we first show that $s_i\mapsto \bar{s}_i$. Then we show that the inclusion map $f: P\longrightarrow \mathbb{X}^n$ induces a $W$-equivariant map which is a homeomorphism. We do this by defining a $\mathbb{X}^n$-structure.


\subsection{The Davis complex}

\begin{definition}
    Let $(W,S)$ be a Coxeter group. The davis complex $\Sigma=\Sigma(W,S)$ of $(W,S)$ is $\mathcal{U}(W,K)$, where  the chamber $K$ has mirror structure $(K_s)_{s\in S}$ such that $\forall x\in K$, $W_{S(x)}$ is finite.
\end{definition}


\begin{definition}
    We say that a subset $T\subseteq S$ is spherical if $W_T=\langle T\rangle$ is finite. In this case, we call $W_T$ a spherical special subgroup.  
\end{definition}

How do we construct $K$? Consider the set 
\[L=\{T\subseteq S|T\neq\emptyset, T\text{ is spherical}\}.\]

Let us note that this set itself forms an abstract simplicial complex. Also note that $\{s\}\in L$ for all $s\in S$. 

\begin{definition}
    This set $L=L(W,S)$ is called the nerve of $(W,S)$. 
\end{definition}

The set $L$ has vertex set $S$, and a simplex $\sigma_T$ spanning each non-empty spherical $T$.

\begin{example}
    Consider a finite Coxeter group $(W,S)$. Then obviously all spherical subgroups of $W$ are also finite. Therefore, the nerve of $W$ is the full simplex on $S$. 
\end{example}

\begin{example}
    
\end{example}


\begin{definition}
    A flag complex is a simplicial complex $L$ such that each finite, non-empty set of vertices $T$ spans a simplex in $L$ if and only if any two elements of $T$ span an edge in $L$. 
\end{definition}

\begin{lemma}
    Consider a right-angled Coxeter system $(W,S)$. Then $L(W,S)$ is a flag complex.
\end{lemma}

\section{Topology of the Davis complex}
\section{Geometry of the Davis complex}
\section{Boundaries of Coxeter groups}


\section{Buildings as apartment systems}

\begin{definition}
    (Tits 1950s) Let $(W,S)$ be a Coxeter group. A building of type $(W,S)$ is a simplicial complex, which is a union of subcomplexes, called apartments. An apartment is a copy of the Coxeter complex for $(W,S)$. The maximal simplicies in the simplicial complex are called chambers. Now the apartments and chambers must satisfy
    \begin{enumerate}
    \item Given any two chambers, there exists an aprtment containing both of them.
    \item Given any two apartments A and B, there exists an isomorphism from A to B which fixes $A \cap B$ pointwise. 
    \end{enumerate}
\end{definition}

We have two descriptions of the Coxeter complex
\begin{enumerate}
    \item It is given by the basic construction $\uw$, where $X$ is the simplex with codimension-one faces $\{\Delta_s|s\in S\}$ and mirrors $X_s=\Delta_s.$ So concretely, $\uw$ is $W$-many copies of $X$, with the s-mirrors wX and wsX glued together. 
    \item It is the geometric realisation of the poset $\{wW_T|T\subseteq S, w\in W\}$, where we order by inclusion.
\end{enumerate}

We also can expand the above definition to let the complex be a poyhedral complex, and allow the apartments to be other geometric realisations of $(W,S)$. 

Specifically, if $(W,S)$ is a geometric reflection group on $\mathbb{X}^n$, then the apartments can be copies of $\mathbb{X}^n$ tiled by copies of $P$ - which are the chambers. 

\begin{example}
    Let us consider a single copy of $X^n$. We can tile this copy by P, and we get a thin building. This means that we only have one apartment. 
\end{example}

\begin{definition}
If the apartment is a Coxeter complex, or more generally a tiling of $\mathbb{X}^n$, a panel is a codimension-one face of a chamber. If the apartment is a Davis complex, a panel is a copy of a mirror.
\end{definition}

\begin{definition}
    A building is thick if every panel is contained in at least three chambers.
\end{definition}


























\end{document}

